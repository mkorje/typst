\documentclass{article}
\usepackage[bidi=basic]{babel}
\babelprovide[import, main, maparabic]{arabic}
\babeladjust{bidi.math=off}
\usepackage{unicode-math}
\babelfont[arabic]{rm}
[
  Extension=.ttf,
  UprightFont=*-Regular,
  BoldFont=*-Bold,
  Renderer=HarfBuzz,
]{NotoSansArabic}
\babelfont[arabic]{sf}
[
  Extension=.ttf,
  UprightFont=*-Regular,
  BoldFont=*-Bold,
  Renderer=HarfBuzz,
]{NotoSansArabic}
\setmathfont{NotoSansMath-Regular}
[
  Extension=.ttf,
  RawFeature=+rtlm,
]

\renewcommand{\baselinestretch}{1.2}
\DeclareMathOperator{\arSin}{\text{جا}}
\DeclareMathOperator{\arCos}{\text{جتا}}
\DeclareMathOperator{\arTan}{\text{قا}}
\DeclareMathOperator{\arCot}{\text{قتا}}
\DeclareMathOperator{\arSec}{\text{ظا}}
\DeclareMathOperator{\arCsc}{\text{ظتا}}

\begin{document}

\mathdir TRT
\parindent 0pt
\pagestyle{empty}

في هذا المثال، نريد إيجاد معدل تغير الإحداثي $𞸑$ لجسيم يتحرك على طول منحنى، كما هو موضح بالشكل، عند اللحظة التي تكون نقطة محددة $)𞸎،𞸑($ ومعطى معدل تغير الإحداثي $𞸎$ ثابتًا.

على الرغم من أن الحل سيكون جبريًا في البداية، فإن الناتج النهائي سيكون عدديًا بعد أن نعوض بالقيم المعلومة.

من أجل إيجاد معدل تغير $𞸑$، $\frac{𞸃𞸑}{𞸃𞸍}$، نبدأ باشتقاق معادلة المنحنى ضمنيًا بالنسبة إلى $𞸍$. بعد ذلك، نعوض بمعدل تغير الإحداثي $𞸎$، $\frac{𞸃𞸎}{𞸃𞸍}=٢$، عند النقطة حيث $𞸎=−١$ و $𞸑=٣$.

إذا أخذنا مشتقة معادلة المنحنى بالنسبة إلى $𞸍$، نجد أن:

\begin{align}
٠	&	=\frac{𞸃}{𞸃𞸍}\left)٦𞸑^٢+٢𞸎^٢−٢𞸎+٥𞸑−٣١\right(	\\
 	&	=٢١𞸑\frac{𞸃𞸑}{𞸃𞸍}+٤𞸎\frac{𞸃𞸎}{𞸃𞸍}−٢\frac{𞸃𞸎}{𞸃𞸍}+٥\frac{𞸃𞸑}{𞸃𞸍}	\\
	&	=\left)٢١𞸑+٥\right(\frac{𞸃𞸑}{𞸃𞸍}+\left)٤𞸎−٢\right(\frac{𞸃𞸎}{𞸃𞸍}.
\end{align}

يمكننا الآن التعويض بالنقطة $𞸎=−١$ و $𞸑=٣$ ومعدل تغير الإحداثي $𞸎$، $\frac{𞸃𞸎}{𞸃𞸍}=٢$:

\begin{align}
٠	&	=\left)٢١×٣+٥\right(\frac{𞸃𞸑}{𞸃𞸍}+\left)٤×−١−٢\right(×٢	\\
	&	=١٤\frac{𞸃𞸑}{𞸃𞸍}−٢١.
\end{align}

ثم بإعادة الترتيب لإيجاد قيمة
$\frac{𞸃𞸑}{𞸃𞸍}$، نحصل على:

\[
\frac{𞸃𞸑}{𞸃𞸍}=\frac{٢١}{١٤}.
\]

إذن، معدل تغير الإحداثي $𞸑$ هو $\frac{٢١}{١٤}$.

%\newpage

باشتقاق المعادلة البارامترية للمتغيِّر $𞸎$ بالنسبة إلى $𝜃$، نحصل على:

\begin{align}
\frac{𞸃𞸎}{𞸃𝜃}	&	=\frac{𞸃}{𞸃𝜃}\left)−٤\arCsc 𝜃+٣\right(	\\
			&	=−٤\frac{𞸃}{𞸃𝜃}\arCsc 𝜃+\frac{𞸃}{𞸃𝜃}\left)٣\right(	\\
		& 	=−٤\left)−\arCot 𝜃\right(+٠	\\
		&	=٤\arCot^٢𝜃.
\end{align}

باشتقاق معادلة $𞸑$، نحصل على:

\begin{align}
\frac{𞸃𞸑}{𞸃𝜃}	&	=\frac{𞸃}{𞸃𝜃} \left)٣\arSin^٢\right(	\\
			&	=٣\frac{𞸃}{𞸃𝜃}\arSin^٢𝜃+\sqrt٢+\frac{𞸃}{𞸃𝜃}+\arTan 𝜃
\end{align}

علينا تطبيق قاعدة السلسلة لإيجاد مشتقة $\arSin^٢ 𝜃$. يمكننا كتابة هذا التعبير على صورة التركيب $د∘𞸓$؛ حيث $د)𝜃(=𝜃^٢$، $𞸓)𝜃(=\arSin 𝜃$. إذن $د')𝜃(=٢𝜃$، $𞸓')𝜃(=\arCos 𝜃$، ما يؤدِّي إلى:

\begin{align}
\frac{𞸃}{𞸃𝜃}\arSin^٢ 𝜃 = د')𞸓)𝜃((𞸓')𝜃(= ٢\arSin ٢𝜃 \arCos 𝜃.
\end{align}

بالتعويض بهذا التعبير وبمشتقة $\arTan 𝜃$ أيضًا في تعبير $\frac{𞸃𞸑}{𞸃𝜃}$، نحصل على:

\begin{align}
\frac{𞸃𞸑}{𞸃𝜃} = ٦\arSin 𝜃 \arCos 𝜃 + \sqrt ٢ \arTan 𝜃 \arSec 𝜃.
\end{align}

ومن ثَمَّ، بتطبيق الاشتقاق البارامتري يكون لدينا:

\begin{align}
\frac{𞸃𞸑}{𞸃𞸎} = \frac{\frac{𞸃𞸑}{𞸃𝜃}}{\frac{𞸃𞸎}{𞸃𝜃}} = \frac{٦\arSin 𝜃 \arCos 𝜃 + \sqrt{٢}\arTan 𝜃 \arSec 𝜃}{٤\arCot^٢𝜃}
\end{align}

وبما أن المستقيم عمودي عند قيمة البارامتر $𝜃=\frac{𝜋}{٤}$، إذن علينا حساب قيمة التعبير السابق عند هذه النقطة:
\begin{align}
 \frac{𞸃𞸑}{𞸃𞸎}\Big|_{𝜃=\frac{𝜋}{٤}}	&	= \frac{\arSin \frac{𝜋}{٤} \arCos\frac{𝜋}{٤} + \sqrt{٢} + \arTan\frac{𝜋}{٤}\arSec\frac{𝜋}{٤}}{٤\arCot^٢\frac{𝜋}{٤}}	\\
								&	= \frac{٦ × \frac{\sqrt ٢}{٢} × \frac{\sqrt ٢}{٢} + \sqrt ٢ × \frac ٢ {\sqrt ٢} × ١}{٤ × \left)\frac ٢{\sqrt ٢}\right(^٢}	\\
								&	= \frac{٣+٢}{٨} \\
								&	= \frac ٥ ٨.
\end{align}
\end{document}

